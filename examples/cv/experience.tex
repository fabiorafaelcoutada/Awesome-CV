%-------------------------------------------------------------------------------
%	SECTION TITLE
%-------------------------------------------------------------------------------
\cvsection{Experience}


%-------------------------------------------------------------------------------
%	CONTENT
%-------------------------------------------------------------------------------
\begin{cventries}

%---------------------------------------------------------
  \cventry
    {Embedded Systems Research Engineer} % Job title
    {Vortex-CoLab} % Organization
    {Porto, Portugal} % Location
    {June 2023 - June 2024} % Date(s)
    {
      \begin{cvitems} % Description(s) of tasks/responsibilities
        \item {Everything that matters, now, in my career.}
        \item {Developed}
        \item {Continuously improved the infrastructure architecture since launching the service. (currently 3.6 million users)}
        \item {Established a standardized base for declarative management of infrastructures and service deployments, enabling operational efficiency and consistency. Over 90\% of AWS resources were all managed through standardized terraform modules. All add-ons and service workloads on the Kubernetes cluster were managed on a GitOps basis with Kustomize and ArgoCD.}
        \item {Saved over 30\% of the overall AWS costs by establishing a quarterly purchasing strategiy for RI (Reserved Instance) and SP (Savings Plan) and by introducing Graviton instances.}
        \item {Established a core architecture for regulating of outbound DNS traffic in multi-account and multi-VPC environments utilizing AWS Route53 DNS Firewall and FMS. This significantly increased the level of security confidence in the financial sector's segregated environment.}
        \item {Introduced Okta employee identity solution in the company, establishing security policies and configuring SSO integration with over 20 enterprise systems including AWS, GitHub, Slack, Google Workspace. Set up a Hub and Spoke architecture, enabling a collaborative account structure with the parent company, Daangn Market.}
      \end{cvitems}
    }

%---------------------------------------------------------
  \cventry
    {Embedded Systems Engineer} % Job title
    {ispace Inc.} % Organization
    {Luxembourg, Luxembourg} % Location
    {Sept. 2023 - May 2024} % Date(s)
    {
        \begin{cvitems} % Description(s) of tasks/responsibilities
            \item {This was a position with a lot of responsibilities, ranging from Embedded Software development and FPGA design to Embedded Linux and Kernel drivers development.}
            \item {As an Embedded Linux Engineer at ispace‑inc, I was responsible for generating firmware for a Xilinx Zynq‑7000 development board. To achieve this, I utilized several key tools and technologies, including Yocto, U‑Boot, and the latest version of the Linux kernel. I also implemented the real‑time patch for the Linux kernel, similar to the one used on NASA’s Perseverance rover.}
            \item {My responsibilities included developing all kernel drivers to manage USB cameras, IIC sensors, and SPI sensors. This involved designing, developing, and testing the drivers to ensure they were fully functional and met project requirements. Another key focus was the development of the flight software for the project. For this task, I used the C programming language and integrated image processing libraries from Xilinx to be included on the FPGA, ensuring the flight software was fully optimized and met performance specifications.}
            \item {To manage the development process, I used GitLab to track changes and collaborate with team members. Additionally, I employed CI/CD with GitLab to automate the build and test process, ensuring the software was always up to date and met quality standards.}
            \item {Overall, my role at ispace‑inc was crucial to the project’s constant change requests. By leveraging my expertise in key tools and technologies and focusing on FPGA development, kernel development, driver development, and flight software development, I contributed to the project’s evaluation phases by ESA, and helped ensure it met the required specifications.}
        \end{cvitems}
    }

%---------------------------------------------------------
  \cventry
    {Embedded Systems Engineer and Tester} % Job title
    {DSR Corp} % Organization
    {Seoul, S.Korea} % Location
    {Aug. 2018 - Aug. 2022} % Date(s)
    {
      \begin{cvitems} % Description(s) of tasks/responsibilities
        \item {This position was a mix between Embedded Software development and testing.}
        \item {During this work placement, I served as an Embedded Software Engineer on the Embedded Systems team, where we were responsible for the certification of a commercial hard real‑time operating system, developed for safety and security‑critical applications.}
        \item {These applications spanned various fields, including Aerospace and Defense, Automotive and Transportation, Industrial Automation and Medical, Network Infrastructures, and Consumer Electronics.}
        \item {The core concept of this RTOS was its ability to safely execute applications with different safety levels concurrently on the same platform. This was achieved by hosting one or more applications inside Virtual Machines, each with specific memory, CPU time, and I/O access rights. These applications ranged from simple control loops to complete paravirtualized guest operating systems like Linux. Essentially, this RTOS functioned as a Type 1 Hypervisor.}
        \item {The project required Continuous Integration development to ensure that all software requirements for the Hypervisor were completely satisfied during runtime testing of the RTOS. This involved correctly configuring the necessary Virtual Machines, Device Drivers, and Communication systems.}
        \item {My responsibilities included developing and testing the software, ensuring it met the required specifications and standards. I also worked on the development of the software architecture, ensuring it was scalable and maintainable.}
        \item {The successful development of the certification work packages relied on several key aspects, including C language development, Linux Host development, and the use of tools such as qemu, gcc, gdb, make, and bash. We used Jira for issue reporting and task tracking, and GitLab for version control and code review. A Test Framework was utilized for all certification tests, which included automated tests, unit tests, and hardware‑in‑loop tests.}
        \item {Additionally, we used SVN for document and planning documentation, and IBM Rational DOORS for requirement creation and traceability. Python and bash were employed for workload automation.}
        \item {Overall, my role at DSR Corp was crucial to the project’s success. By leveraging my expertise in key tools and technologies and focusing on software development and testing, I contributed to the project’s evaluation phases by the certification authority, and helped ensure it met the required specifications.}
      \end{cvitems}
    }

%---------------------------------------------------------
  \cventry
    {Embedded Linux Engineer} % Job title
    {Altice Labs} % Organization
    {Aveiro, Portugal} % Location
    {Jan. 2018 - Aug. 2018} % Date(s)
    {
      \begin{cvitems} % Description(s) of tasks/responsibilities
        \item {This was a freelancer position.}
        \item {I was responsible for developing a custom Linux Kernel driver for a PCIe device, that was used by an application responsible with the flow controls of a fiber optic network.}
        \item {The driver was developed in C and was responsible for the communication between the PCIe device and the user space application.}
        \item {The Device Driver was for a PCI‑express interface between a Cortex‑A53 processor and the FPGA area present in the development board.}
      \end{cvitems}
    }

%---------------------------------------------------------
  \cventry
    {Embedded Systems Engineer} % Job title
    {CTAG (Renault-Nissan)} % Organization
    {Pontevedra, Spain} % Location
    {Jan. 2017 - Dec. 2017} % Date(s)
    {
      \begin{cvitems} % Description(s) of tasks/responsibilities
        \item {The objective was the automation of door handles and ADAS cameras for a Renault‑Nissan electric vehicle.}
        \item {The project was developed in C and C++ and was responsible for the communication between the door handles and the ADAS cameras with the vehicle's main computer.}
        \item {The firmware was using a bare‑metal approach, the microprocessor was a Cortex‑M4, and the HAL used IIC, UART, CAN bus, and LIN protocol.}
      \end{cvitems}
    }

%---------------------------------------------------------
  \cventry
    {Academic Researcher} % Job title
    {Master’s Thesis, Embedded Systems (Prof. Adriano Tavares)} % Organization
    {Guimaraes, Portugal} % Location
    {Mar. 2016 - Dec. 2016} % Date(s)
    {
      \begin{cvitems} % Description(s) of tasks/responsibilities
        \item {Internet of Things systems, based in mbed OS, running on Microsemi Smartfusion2 SoC.}
        \item {Responsible for analyzing and managing all the stages of research and development of a Wireless Sensor Network using LTE Cat‑M, Bluetooth 5 and WiFi connectivity.}
        \item {The purpose of this project was the implementation of an embedded system that would be used in Precision Agriculture.}
        \item {The system has various smart sensors used to obtain data (Ionic concentration of nutrients, temperature, CO2, humidity, wind speed and light intensity) that is sent using Bluetooth to the central nodes,responsible to analyze and upload the data to the Cloud using either WiFi or 4G depending on the central deployment in the field.}
        \item {The main objective is to have a database to later be used in Data Analytics and Machine Learning, in order to improve the efficiency of the overall production.}
        \item {The project created the need to port mbed OS to the Smartfusion2 SoC, running mbed OS, and to develop the drivers for the sensors and the communication modules.}
        \item {The main research part of the thesis, would be the development of a Machine Learning accelerator, or more specifically, and Hardware thread, inside the FPGA area, that would be used to accelerate the Machine Learning algorithms used in the data analysis.}
      \end{cvitems}
    }

%---------------------------------------------------------
  \cventry
    {Academic Researcher} % Job title
    {Master’s Research project, Embedded Systems Lab (Prof. Adriano Tavares)} % Organization
    {Guimaraes, Portugal} % Location
    {Sept. 2015 - Feb. 2016} % Date(s)
    {
      \begin{cvitems} % Description(s) of tasks/responsibilities
        \item {Radio‑Frequency Jamming Attack on GSM and Wireless Networks.}
        \item {Responsible for analyzing and managing all the stages of research and development of a RF Jammer circuit to be used to jam GSM and WiFi networks in a 3‑meter radius.}
        \item {The project was developed using the C language, the processor used was Xilinx Zynq‑7000 SoC}
        \item {One of the main requirements was the use of Embedded Linux (Buildroot) and Multithreading (Posix) to develop the SW application that controls the HW part of the project.}
        \item {The project was also combined a HMI with touchscreen interface, developed using Qt Creator.}
        \item {All the schematic design was done using Altium PCB Designer.}
      \end{cvitems}
    }
%---------------------------------------------------------
\end{cventries}
